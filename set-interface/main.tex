\chapter{Set Interface}
\label{ch:set-interface}

\begin{preamble}
The \sml{SET} interface specifies an unordered collection of items. Sets do not
contain duplicates, and are not polymorphic: the type of their elements is
given by the \sml{Key} substructure.
\end{preamble}

\section{Summary}
\begin{gram}
\begin{verbatim}
signature SET =
sig
  structure Key : EQKEY
  structure Seq : SEQUENCE

  type t
  type set = t

  val size : set -> int
  val toString : set -> string
  val toSeq : set -> Key.t Seq.t

  val empty : unit -> set
  val singleton : Key.t -> set
  val fromSeq : Key.t Seq.t -> set

  val find : set -> Key.t -> bool
  val insert : set * Key.t -> set
  val delete : set * Key.t -> set

  val filterKey : (Key.t -> bool) -> set -> set

  val reduceKey : (Key.t * Key.t -> Key.t) -> Key.t -> set -> Key.t
  val iterateKey : ('a * Key.t -> 'a) -> 'a -> set -> 'a

  val union : set * set -> set
  val intersection : set * set -> set
  val difference : set * set -> set

  val $ : Key.t -> set
end
\end{verbatim}
\end{gram}


\section{Substructures}

\begin{gram}
\begin{verbatim}
structure Key : EQKEY
\end{verbatim}
The \sml{Key} substructure defines the type of elements in a set, which may be
compared for equality.
\end{gram}

\begin{gram}
\begin{verbatim}
structure Seq : SEQUENCE
\end{verbatim}
The \sml{Seq} substructure defines the underlying sequence type, so that we
may convert sets to and from sequences.
\end{gram}


\section{Types}

\begin{gram}
\begin{verbatim}
type t
type set = t
\end{verbatim}
The abstract set type. The alias \sml{set} is for readability in the
signature.
\end{gram}


\section{Values}

\begin{gram}[size]
\begin{verbatim}
val size : set -> int
\end{verbatim}
The number of elements in a set.
\end{gram}

\begin{gram}[toString]
\begin{verbatim}
val toString : set -> string
\end{verbatim}
Evaluates to a string representation of the set. Each element is converted to
a string via \sml{Key.toString}.
\end{gram}

\begin{gram}[toSeq]
\begin{verbatim}
val toSeq : set -> Key.t Seq.t
\end{verbatim}
Return a sequence of all keys in a set. The ordering of the elements in the
returned sequence is implementation-defined.
\end{gram}

\begin{gram}[empty]
\begin{verbatim}
val empty : unit -> set
\end{verbatim}
Construct the empty set.
\end{gram}

\begin{gram}[singleton]
\label{gr:set-interface:singleton}
\begin{verbatim}
val singleton : Key.t -> set
\end{verbatim}
Construct the singleton set containing only the provided key.
\end{gram}

\begin{gram}[fromSeq]
\begin{verbatim}
val fromSeq : Key.t Seq.t -> set
\end{verbatim}
Return the set of all elements of a sequence.
\end{gram}

\begin{gram}[find]
\begin{verbatim}
val find : set -> Key.t -> bool
\end{verbatim}
\sml{find s k} returns whether or not $k$ is a member of the set $s$.
\end{gram}

\begin{gram}[insert]
\begin{verbatim}
val insert : set * Key.t -> set
\end{verbatim}
\sml{insert (s, k)} evaluates to the set $s \cup \{k\}$.
\end{gram}

\begin{gram}[delete]
\begin{verbatim}
val delete : set * Key.t -> set
\end{verbatim}
\sml{delete (s, k)} evaluates to the set $s \setminus \{k\}$.
\end{gram}

\begin{gram}[filterKey]
\begin{verbatim}
val filterKey : (Key.t -> bool) -> set -> set
\end{verbatim}
\sml{filterKey p s} evaluates to the set $\{x \in s \mathbin| p(x) \}$
\end{gram}

\begin{gram}[reduceKey]
\begin{verbatim}
val reduceKey : (Key.t * Key.t -> Key.t) -> Key.t -> set -> Key.t
\end{verbatim}
\sml{reduceKey f b s} is logically equivalent to \sml{Seq.reduce f b (toSeq s)}.
\end{gram}

\begin{gram}[iterateKey]
\begin{verbatim}
val iterateKey : ('a * Key.t -> 'a) -> 'a -> set -> 'a
\end{verbatim}
\sml{iterateKey f b s} is logically equivalent to \sml{Seq.iterate f b (toSeq s)}.
\end{gram}

\begin{gram}[union]
\begin{verbatim}
val union : set * set -> set
\end{verbatim}
\sml{union a b} evaluates to the set $a \cup b$.
\end{gram}

\begin{gram}[intersection]
\begin{verbatim}
val intersection : set * set -> set
\end{verbatim}
\sml{intersection a b} evaluates to the set $a \cap b$.
\end{gram}

\begin{gram}[difference]
\begin{verbatim}
val difference : set * set -> set
\end{verbatim}
\sml{difference a b} evaluates to the set $a \setminus b$.
\end{gram}

\begin{gram}[\$]
\begin{verbatim}
val $ : Key.t -> set
\end{verbatim}
An alias for \ref{gr:set-interface:singleton}.
\end{gram}


