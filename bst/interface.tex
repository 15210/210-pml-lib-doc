\chapter{BST Interface}
\label{ch:bst-interface}
\begin{preamble}
The \sml{BST} signature is a minimalistic interface for a binary search tree, which
stores a set of key-value pairs. The key type is ordered and fixed by the
\sml{Key} substructure while the value type is polymorphic.
\end{preamble}

\section{Summary}
\begin{gram}
\begin{verbatim}
signature BST =
sig
  structure Key : ORDKEY

  type 'a t
  type 'a bst = 'a t

  datatype 'a view =
    LEAF
  | NODE of { key : Key.t
            , value : 'a
            , left : 'a bst
            , right : 'a bst }

  exception Order

  val expose : 'a bst -> 'a view
  val size : 'a bst -> int

  val empty : unit -> 'a bst
  val singleton : Key.t * 'a -> 'a bst

  val join : 'a bst * 'a bst -> 'a bst
  val joinMid : 'a bst * (Key.t * 'a) * 'a bst -> 'a bst

  val split : 'a bst * Key.t -> 'a bst * 'a option * 'a bst

  val $ : Key.t * 'a -> 'a bst
end
\end{verbatim}
\end{gram}

\section{Types}

\begin{gram}
\begin{verbatim}
type 'a t
type 'a bst = 'a t
\end{verbatim}
The abstract tree type \sml{'a t} maps keys of type \sml{Key.t} to values of
type \sml{'a}. The alias \sml{'a bst} is for readability.
\end{gram}

\begin{gram}
\begin{verbatim}
datatype 'a view = LEAF | NODE of {key : Key.t, value : 'a, left : 'a bst, right : 'a bst}
\end{verbatim}
A one-layer view of a tree, which is either a leaf (containing no data) or an
interior node with a key, value, and two children.
\end{gram}

\section{Exceptions}

\begin{gram}
\begin{verbatim}
exception Order
\end{verbatim}
\sml{Order} is raised when the ordering invariant would be violated.
\end{gram}


\section{Values}

\begin{gram}[expose]
\begin{verbatim}
val expose : 'a bst -> 'a view
\end{verbatim}
View the root node of the tree.
\end{gram}

\begin{gram}[size]
\begin{verbatim}
val size : 'a bst -> int
\end{verbatim}
Return the number of nodes in the tree.
\end{gram}

\begin{gram}[empty]
\begin{verbatim}
val empty : unit -> 'a bst
\end{verbatim}
Construct an empty tree.
\end{gram}

\begin{gram}[singleton]
\begin{verbatim}
val singleton : Key.t * 'a -> 'a bst
\end{verbatim}
Construct a tree containing a single node with the given key and value.
\end{gram}

\begin{gram}[join]
\begin{verbatim}
val join : 'a bst * 'a bst -> 'a bst
\end{verbatim}
When given $A$ and $B$ where the largest key in $A$ is strictly smaller than
the smallest key in $B$, \sml{join(A,B)} is the union of all key-value pairs
from both $A$ and $B$. Raises \sml{Order} otherwise.
\end{gram}

\begin{gram}[joinMid]
\begin{verbatim}
val joinMid : 'a bst * (Key.t * 'a) * 'a bst -> 'a bst
\end{verbatim}
\sml{joinMid(A, (k,v), B)} is logically equivalent to
\sml{join(A, join(singleton(k,v), B))}.
\end{gram}

\begin{gram}[split]
\begin{verbatim}
val split : 'a bst * Key.t -> 'a bst * 'a option * 'a bst
\end{verbatim}
\sml{split(T,k)} evaluates to $(L,v,R)$ where $L$ contains all keys from $T$
which are smaller than $k$, $R$ contains all keys larger than
$k$, and $v$ is the value associated with $k$ (or \sml{NONE} if $k$ is not
in $T$).
\end{gram}

\begin{gram}[\$]
\begin{verbatim}
val $ : Key.t * 'a -> 'a bst
\end{verbatim}
An alias for singleton.
\end{gram}

