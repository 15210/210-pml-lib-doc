\chapter{Single-Threaded Sequence Interface}
\label{ch:st-seq-interface}
\begin{preamble}
The \sml{ST\_SEQUENCE} signature is a minimalistic interface for a
single-threaded sequence type. Single-threaded sequences differ from normal
sequences in that they are only meant to be used in a single-threaded
manner: updates should only be made on a ``most recent'' version. However,
operations on single-threaded sequences are always well-defined, regardless of
context.
\end{preamble}

\section{Summary}
\begin{gram}
\begin{verbatim}
signature ST_SEQUENCE =
sig
  structure Seq : SEQUENCE

  type 'a t
  type 'a stseq = 'a t

  exception Range

  val fromSeq : 'a Seq.t -> 'a stseq
  val toSeq : 'a stseq -> 'a Seq.t
  val nth : 'a stseq -> int -> 'a

  val update : ('a stseq * (int * 'a)) -> 'a stseq
  val inject : ('a stseq * (int * 'a) Seq.t) -> 'a stseq
end
\end{verbatim}
\end{gram}

\section{Substructures}

\begin{gram}[Seq]
\label{gr:st-seq-interface:Seq}
\begin{verbatim}
structure Seq : SEQUENCE
\end{verbatim}
Defines the underlying sequence type to and from which single-threaded
sequences can be converted.
\end{gram}


\section{Types}

\begin{gram}
\begin{verbatim}
type 'a t
type 'a stseq = 'a t
\end{verbatim}
The abstract single-threaded sequence type \sml{'a t} has elements of type
\sml{'a}. The alias \sml{'a stseq} is for readability.
\end{gram}


\section{Exceptions}

\begin{gram}
\begin{verbatim}
exception Range
\end{verbatim}
The \sml{Range} exception is raised whenever an invalid index into a sequence
is used.
\end{gram}


\section{Values}

\begin{gram}[fromSeq]
\label{gr:st-seq-interface:fromSeq}
\begin{verbatim}
val fromSeq : 'a Seq.t -> 'a stseq
\end{verbatim}
Convert a sequence to a single-threaded sequence.
\end{gram}

\begin{gram}[toSeq]
\label{gr:st-seq-interface:toSeq}
\begin{verbatim}
val toSeq : 'a stseq -> 'a Seq.t
\end{verbatim}
Convert a single-threaded sequence to a standard sequence.
\end{gram}

\begin{gram}[nth]
\label{gr:st-seq-interface:nth}
\begin{verbatim}
val nth : 'a stseq -> int -> 'a
\end{verbatim}
\sml{nth s i} returns the $i^\text{th}$ element of $s$. Raises \sml{Range}
if $i$ is out-of-bounds.
\end{gram}

\begin{gram}[update]
\label{gr:st-seq-interface:update}
\begin{verbatim}
val update : ('a stseq * (int * 'a)) -> 'a stseq
\end{verbatim}
\sml{update (s, (i, x))} evaluates to a new single-threaded sequence where
the $i^\text{th}$ element of $s$ has been replaced by $x$, but all other
elements are the same as in $s$. Raises \sml{Range} if $i$ is out-of-bounds.
\end{gram}

\begin{gram}[inject]
\label{gr:st-seq-interface:inject}
\begin{verbatim}
val inject : ('a stseq * (int * 'a) Seq.t) -> 'a stseq
\end{verbatim}
\sml{inject (s, u)} evaluates to a new single-threaded sequence where, for
each $(i,x) \in u$, the $i^\text{th}$ element of $s$ has been replaced by $x$,
but all other elements are the same as in $s$. If there are duplicate indices
in $u$, one of them ``wins'' non-deterministically, and the others are ignored.
Raises \sml{Range} if any of the indices in $u$ is out-of-bounds.
\end{gram}

