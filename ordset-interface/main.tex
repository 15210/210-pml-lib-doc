\chapter{OrdSet Interface}
\label{ch:ordset-interface}
\begin{preamble}
The \sml{ORDSET} interface specifies an ordered collection of items. These sets
do not contain duplicates, and are not polymorphic: the type of their
elements is given by the \sml{Key} substructure.
\end{preamble}

\begin{note}
The \sml{ORDSET} signature is identical to \sml{SET} except for the following.
\begin{itemize}
  \item The \sml{Key} substructure now ascribes to \sml{ORDKEY}.
  \item There is additional functionality relying upon the ordering of keys,
  such as \sml{split}, \sml{join}, and \sml{getRange}.
  \item The \sml{toSeq} function now specifies that it returns keys in sorted order.
\end{itemize}
\end{note}

\section{Summary}
\begin{gram}
\begin{verbatim}
signature ORDSET =
sig
  structure Key : ORDKEY
  structure Seq : SEQUENCE

  type t
  type set = t

  exception Order

  val size : set -> int
  val toString : set -> string
  val toSeq : set -> Key.t Seq.t

  val empty : unit -> set
  val singleton : Key.t -> set
  val fromSeq : Key.t Seq.t -> set

  val find : set -> Key.t -> bool
  val insert : set * Key.t -> set
  val delete : set * Key.t -> set

  val filterKey : (Key.t -> bool) -> set -> set

  val reduceKey : (Key.t * Key.t -> Key.t) -> Key.t -> set -> Key.t
  val iterateKey : ('a * Key.t -> 'a) -> 'a -> set -> 'a

  val union : set * set -> set
  val intersection : set * set -> set
  val difference : set * set -> set

  val $ : Key.t -> set

  (* ordered sets begins here *)
  val first : set -> Key.t option
  val last : set -> Key.t option

  val prev : set * Key.t -> Key.t option
  val next : set * Key.t -> Key.t option

  val split : set * Key.t -> set * bool * set
  val join : set * set -> set

  val getRange : set -> Key.t * Key.t -> set

  val rank : set * Key.t -> int
  val select : set * int -> Key.t option
  val splitRank : set * int -> set * set
end
\end{verbatim}
\end{gram}

\section{Substructures}

\begin{gram}
\begin{verbatim}
structure Key : ORDKEY
\end{verbatim}
The \sml{Key} substructure defines the type of elements in a set, which are
totally ordered according to the provided comparison function.
\end{gram}

\begin{gram}
\begin{verbatim}
structure Seq : SEQUENCE
\end{verbatim}
The \sml{Seq} substructure defines a sequence type for use with
\ref{gr:ordset-interface:toSeq} and \ref{gr:ordset-interface:fromSeq}.
\end{gram}


\section{Types}

\begin{gram}
\begin{verbatim}
type t
type set = t
\end{verbatim}
The abstract set type. The alias \sml{set} is for readability in the
signature.
\end{gram}


\section{Exceptions}

\begin{gram}
\begin{verbatim}
exception Order
\end{verbatim}
\sml{Order} is raised when the ordering invariant would be violated.
\end{gram}


\section{Values}

\begin{gram}[size]
\begin{verbatim}
val size : set -> int
\end{verbatim}
The number of elements in a set.
\end{gram}

\begin{gram}[toString]
\begin{verbatim}
val toString : set -> string
\end{verbatim}
Evaluates to a string representation of the set. Each element is converted to
a string via \sml{Key.toString}.
\end{gram}

\begin{gram}[toSeq]
\label{gr:ordset-interface:toSeq}
\begin{verbatim}
val toSeq : set -> Key.t Seq.t
\end{verbatim}
Return a sequence of all keys in a set, in sorted order.
\end{gram}

\begin{gram}[empty]
\begin{verbatim}
val empty : unit -> set
\end{verbatim}
Construct the empty set.
\end{gram}

\begin{gram}[singleton]
\label{gr:ordset-interface:singleton}
\begin{verbatim}
val singleton : Key.t -> set
\end{verbatim}
Construct the singleton set containing only the provided key.
\end{gram}

\begin{gram}[fromSeq]
\label{gr:ordset-interface:fromSeq}
\begin{verbatim}
val fromSeq : Key.t Seq.t -> set
\end{verbatim}
Return the set of all elements of a sequence.
\end{gram}

\begin{gram}[find]
\begin{verbatim}
val find : set -> Key.t -> bool
\end{verbatim}
\sml{find s k} returns whether or not $k$ is a member of the set $s$.
\end{gram}

\begin{gram}[insert]
\begin{verbatim}
val insert : set * Key.t -> set
\end{verbatim}
\sml{insert (s, k)} evaluates to the set $s \cup \{k\}$.
\end{gram}

\begin{gram}[delete]
\begin{verbatim}
val delete : set * Key.t -> set
\end{verbatim}
\sml{delete (s, k)} evaluates to the set $s \setminus \{k\}$.
\end{gram}

\begin{gram}[filterKey]
\begin{verbatim}
val filterKey : (Key.t -> bool) -> set -> set
\end{verbatim}
\sml{filterKey p s} evaluates to the set $\{x \in s \mathbin| p(x) \}$
\end{gram}

\begin{gram}[reduceKey]
\begin{verbatim}
val reduceKey : (Key.t * Key.t -> Key.t) -> Key.t -> set -> Key.t
\end{verbatim}
\sml{reduceKey f b s} is logically equivalent to \sml{Seq.reduce f b (toSeq s)}.
\end{gram}

\begin{gram}[iterateKey]
\begin{verbatim}
val iterateKey : ('a * Key.t -> 'a) -> 'a -> set -> 'a
\end{verbatim}
\sml{iterateKey f b s} is logically equivalent to \sml{Seq.iterate f b (toSeq s)}.
\end{gram}

\begin{gram}[union]
\begin{verbatim}
val union : set * set -> set
\end{verbatim}
\sml{union a b} evaluates to the set $a \cup b$.
\end{gram}

\begin{gram}[intersection]
\begin{verbatim}
val intersection : set * set -> set
\end{verbatim}
\sml{intersection a b} evaluates to the set $a \cap b$.
\end{gram}

\begin{gram}[difference]
\begin{verbatim}
val difference : set * set -> set
\end{verbatim}
\sml{difference a b} evaluates to the set $a \setminus b$.
\end{gram}

\begin{gram}[\$]
\begin{verbatim}
val $ : Key.t -> set
\end{verbatim}
An alias for \ref{gr:ordset-interface:singleton}.
\end{gram}

\begin{gram}[first]
\begin{verbatim}
val first : set -> Key.t option
\end{verbatim}
Return the least element of the set, or \sml{NONE} if the set is empty.
\end{gram}

\begin{gram}[last]
\begin{verbatim}
val last : set -> Key.t option
\end{verbatim}
Return the greatest element of the set, or \sml{NONE} if the set is empty.
\end{gram}

\begin{gram}[prev]
\begin{verbatim}
val prev : set * Key.t -> Key.t option
\end{verbatim}
\sml{prev (s, k)} evaluates to $\max \{ k' \in s \mathbin| k' < k \}$, or
\sml{NONE} if there is no such element.
\end{gram}

\begin{gram}[next]
\begin{verbatim}
val next : set * Key.t -> Key.t option
\end{verbatim}
\sml{next (s, k)} evaluates to $\min \{ k' \in s \mathbin| k' > k \}$, or
\sml{NONE} if there is no such element.
\end{gram}

\begin{gram}[split]
\begin{verbatim}
val split : set * Key.t -> set * bool * set
\end{verbatim}
\sml{split (s, k)} evaluates to $(\ell, m, r)$ where $\ell$ is the set of elements
in $s$ which are smaller than $k$, $r$ is the set of elements in $s$ which
are greater than $k$, and $m$ indicates whether or not $k$ is a member of $s$.
\end{gram}

\begin{gram}[join]
\begin{verbatim}
val join : set * set -> set
\end{verbatim}
For sets $a$ and $b$ where the largest element of $a$ is strictly smaller than
the smallest element of $b$, \sml{join (a, b)} evaluates to $a \cup b$.
Otherwise it raises \sml{Order}.
\end{gram}

\begin{gram}[getRange]
\begin{verbatim}
val getRange : set -> Key.t * Key.t -> set
\end{verbatim}
\sml{getRange s (x, y)} evaluates to the set
$\{ k \in s \mathbin| x \leq k \leq y \}$
\end{gram}

\begin{gram}[rank]
\begin{verbatim}
val rank : set * Key.t -> int
\end{verbatim}
\sml{rank (s, k)} evaluates to $\left| \{ k' \in s \mathbin| k' < k \}\right|$,
the number of elements in $s$ which are strictly smaller than $k$.
\end{gram}

\begin{gram}[select]
\begin{verbatim}
val select : set * int -> Key.t option
\end{verbatim}
\sml{select (s, i)} returns the $i^\text{th}$ smallest element of $s$, or
\sml{NONE} if either $i < 0$ or $i \geq |s|$.
\end{gram}

\begin{gram}[splitRank]
\begin{verbatim}
val splitRank : set * int -> set * set
\end{verbatim}
\sml{splitRank (s, i)} evalutes to $(\ell,r)$ where $\ell$ is the set of the
$i$ smallest elements of $s$, and $r$ is the set of the $|s|-i$ largest
elements of $s$. Raises \sml{Fail} if either $i < 0$ or $i \geq |s|$.
\end{gram}


