\chapter{OrdSet Interface}
\label{ch:ordset-interface}

\begin{cluster}
\label{grp:grm:ordset-interface::ordset}

\begin{gram}
\label{grm:ordset-interface::ordset}
\label{ch:ordset-interface}
\begin{preamble}
The \sml{ORDSET} interface specifies an ordered collection of items. These sets
do not contain duplicates, and are not polymorphic: the type of their
elements is given by the \sml{Key} substructure.
\end{preamble}

\end{gram}
\end{cluster}

\begin{cluster}
\label{grp:nt:ordset-interface::ordset}

\begin{note}
\label{nt:ordset-interface::ordset}
The \sml{ORDSET} signature is identical to \sml{SET} except for the following.
\begin{itemize}
  \item The \sml{Key} substructure now ascribes to \sml{ORDKEY}.
  \item There is additional functionality relying upon the ordering of keys,
  such as \sml{split}, \sml{join}, and \sml{getRange}.
  \item The \sml{toSeq} function now specifies that it returns keys in sorted order.
\end{itemize}

\end{note}
\end{cluster}


\section{Summary}
\label{sec:ordset-interface::summary}

\begin{cluster}
\label{grp:grm:ordset-interface::signature}

\begin{gram}
\label{grm:ordset-interface::signature}
\begin{verbatim}
signature ORDSET =
sig
  structure Key : ORDKEY
  structure Seq : SEQUENCE

  type t
  type set = t

  exception Order

  val size : set -> int
  val toString : set -> string
  val toSeq : set -> Key.t Seq.t

  val empty : unit -> set
  val singleton : Key.t -> set
  val fromSeq : Key.t Seq.t -> set

  val find : set -> Key.t -> bool
  val insert : set * Key.t -> set
  val delete : set * Key.t -> set

  val filterKey : (Key.t -> bool) -> set -> set

  val reduceKey : (Key.t * Key.t -> Key.t) -> Key.t -> set -> Key.t
  val iterateKey : ('a * Key.t -> 'a) -> 'a -> set -> 'a

  val union : set * set -> set
  val intersection : set * set -> set
  val difference : set * set -> set

  val $ : Key.t -> set

  (* ordered sets begins here *)
  val first : set -> Key.t option
  val last : set -> Key.t option

  val prev : set * Key.t -> Key.t option
  val next : set * Key.t -> Key.t option

  val split : set * Key.t -> set * bool * set
  val join : set * set -> set

  val getRange : set -> Key.t * Key.t -> set

  val rank : set * Key.t -> int
  val select : set * int -> Key.t option
  val splitRank : set * int -> set * set
end
\end{verbatim}

\end{gram}
\end{cluster}


\section{Substructures}
\label{sec:ordset-interface::substructures}

\begin{cluster}
\label{grp:grm:ordset-interface::structure}

\begin{gram}
\label{grm:ordset-interface::structure}
\begin{verbatim}
structure Key : ORDKEY
\end{verbatim}
The \sml{Key} substructure defines the type of elements in a set, which are
totally ordered according to the provided comparison function.

\end{gram}
\end{cluster}

\begin{cluster}
\label{grp:grm:ordset-interface::sequence}

\begin{gram}
\label{grm:ordset-interface::sequence}
\begin{verbatim}
structure Seq : SEQUENCE
\end{verbatim}
The \sml{Seq} substructure defines a sequence type for use with
\ref{gr:ordset-interface:toSeq} and \ref{gr:ordset-interface:fromSeq}.

\end{gram}
\end{cluster}


\section{Types}
\label{sec:ordset-interface::types}

\begin{cluster}
\label{grp:grm:ordset-interface::type}

\begin{gram}
\label{grm:ordset-interface::type}
\begin{verbatim}
type t
type set = t
\end{verbatim}
The abstract set type. The alias \sml{set} is for readability in the
signature.

\end{gram}
\end{cluster}


\section{Exceptions}
\label{sec:ordset-interface::exceptions}

\begin{cluster}
\label{grp:grm:ordset-interface::exception}

\begin{gram}
\label{grm:ordset-interface::exception}
\begin{verbatim}
exception Order
\end{verbatim}
\sml{Order} is raised when the ordering invariant would be violated.

\end{gram}
\end{cluster}


\section{Values}
\label{sec:ordset-interface::values}

\begin{cluster}
\label{grp:grm:ordset-interface::size}

\begin{gram}[size]
\label{grm:ordset-interface::size}
\begin{verbatim}
val size : set -> int
\end{verbatim}
The number of elements in a set.

\end{gram}
\end{cluster}

\begin{cluster}
\label{grp:grm:ordset-interface::tostring}

\begin{gram}[toString]
\label{grm:ordset-interface::tostring}
\begin{verbatim}
val toString : set -> string
\end{verbatim}
Evaluates to a string representation of the set. Each element is converted to
a string via \sml{Key.toString}.

\end{gram}
\end{cluster}

\begin{cluster}
\label{grp:gr:ordset-interface:toSeq}

\begin{gram}[toSeq]
\label{gr:ordset-interface:toSeq}
\begin{verbatim}
val toSeq : set -> Key.t Seq.t
\end{verbatim}
Return a sequence of all keys in a set, in sorted order.

\end{gram}
\end{cluster}

\begin{cluster}
\label{grp:grm:ordset-interface::empty}

\begin{gram}[empty]
\label{grm:ordset-interface::empty}
\begin{verbatim}
val empty : unit -> set
\end{verbatim}
Construct the empty set.

\end{gram}
\end{cluster}

\begin{cluster}
\label{grp:gr:ordset-interface:singleton}

\begin{gram}[singleton]
\label{gr:ordset-interface:singleton}
\begin{verbatim}
val singleton : Key.t -> set
\end{verbatim}
Construct the singleton set containing only the provided key.

\end{gram}
\end{cluster}

\begin{cluster}
\label{grp:gr:ordset-interface:fromSeq}

\begin{gram}[fromSeq]
\label{gr:ordset-interface:fromSeq}
\begin{verbatim}
val fromSeq : Key.t Seq.t -> set
\end{verbatim}
Return the set of all elements of a sequence.

\end{gram}
\end{cluster}

\begin{cluster}
\label{grp:grm:ordset-interface::find}

\begin{gram}[find]
\label{grm:ordset-interface::find}
\begin{verbatim}
val find : set -> Key.t -> bool
\end{verbatim}
\sml{find s k} returns whether or not $k$ is a member of the set $s$.

\end{gram}
\end{cluster}

\begin{cluster}
\label{grp:grm:ordset-interface::insert}

\begin{gram}[insert]
\label{grm:ordset-interface::insert}
\begin{verbatim}
val insert : set * Key.t -> set
\end{verbatim}
\sml{insert (s, k)} evaluates to the set $s \cup \{k\}$.

\end{gram}
\end{cluster}

\begin{cluster}
\label{grp:grm:ordset-interface::delete}

\begin{gram}[delete]
\label{grm:ordset-interface::delete}
\begin{verbatim}
val delete : set * Key.t -> set
\end{verbatim}
\sml{delete (s, k)} evaluates to the set $s \setminus \{k\}$.

\end{gram}
\end{cluster}

\begin{cluster}
\label{grp:grm:ordset-interface::filterkey}

\begin{gram}[filterKey]
\label{grm:ordset-interface::filterkey}
\begin{verbatim}
val filterKey : (Key.t -> bool) -> set -> set
\end{verbatim}
\sml{filterKey p s} evaluates to the set $\{x \in s \mathbin| p(x) \}$

\end{gram}
\end{cluster}

\begin{cluster}
\label{grp:grm:ordset-interface::reducekey}

\begin{gram}[reduceKey]
\label{grm:ordset-interface::reducekey}
\begin{verbatim}
val reduceKey : (Key.t * Key.t -> Key.t) -> Key.t -> set -> Key.t
\end{verbatim}
\sml{reduceKey f b s} is logically equivalent to \sml{Seq.reduce f b (toSeq s)}.

\end{gram}
\end{cluster}

\begin{cluster}
\label{grp:grm:ordset-interface::iteratekey}

\begin{gram}[iterateKey]
\label{grm:ordset-interface::iteratekey}
\begin{verbatim}
val iterateKey : ('a * Key.t -> 'a) -> 'a -> set -> 'a
\end{verbatim}
\sml{iterateKey f b s} is logically equivalent to \sml{Seq.iterate f b (toSeq s)}.

\end{gram}
\end{cluster}

\begin{cluster}
\label{grp:grm:ordset-interface::union}

\begin{gram}[union]
\label{grm:ordset-interface::union}
\begin{verbatim}
val union : set * set -> set
\end{verbatim}
\sml{union a b} evaluates to the set $a \cup b$.

\end{gram}
\end{cluster}

\begin{cluster}
\label{grp:grm:ordset-interface::intersection}

\begin{gram}[intersection]
\label{grm:ordset-interface::intersection}
\begin{verbatim}
val intersection : set * set -> set
\end{verbatim}
\sml{intersection a b} evaluates to the set $a \cap b$.

\end{gram}
\end{cluster}

\begin{cluster}
\label{grp:grm:ordset-interface::difference}

\begin{gram}[difference]
\label{grm:ordset-interface::difference}
\begin{verbatim}
val difference : set * set -> set
\end{verbatim}
\sml{difference a b} evaluates to the set $a \setminus b$.

\end{gram}
\end{cluster}

\begin{cluster}
\label{grp:grm:ordset-interface::alias}

\begin{gram}[\$]
\label{grm:ordset-interface::alias}
\begin{verbatim}
val $ : Key.t -> set
\end{verbatim}
An alias for \ref{gr:ordset-interface:singleton}.

\end{gram}
\end{cluster}

\begin{cluster}
\label{grp:grm:ordset-interface::first}

\begin{gram}[first]
\label{grm:ordset-interface::first}
\begin{verbatim}
val first : set -> Key.t option
\end{verbatim}
Return the least element of the set, or \sml{NONE} if the set is empty.

\end{gram}
\end{cluster}

\begin{cluster}
\label{grp:grm:ordset-interface::last}

\begin{gram}[last]
\label{grm:ordset-interface::last}
\begin{verbatim}
val last : set -> Key.t option
\end{verbatim}
Return the greatest element of the set, or \sml{NONE} if the set is empty.

\end{gram}
\end{cluster}

\begin{cluster}
\label{grp:grm:ordset-interface::prev}

\begin{gram}[prev]
\label{grm:ordset-interface::prev}
\begin{verbatim}
val prev : set * Key.t -> Key.t option
\end{verbatim}
\sml{prev (s, k)} evaluates to $\max \{ k' \in s \mathbin| k' < k \}$, or
\sml{NONE} if there is no such element.

\end{gram}
\end{cluster}

\begin{cluster}
\label{grp:grm:ordset-interface::next}

\begin{gram}[next]
\label{grm:ordset-interface::next}
\begin{verbatim}
val next : set * Key.t -> Key.t option
\end{verbatim}
\sml{next (s, k)} evaluates to $\min \{ k' \in s \mathbin| k' > k \}$, or
\sml{NONE} if there is no such element.

\end{gram}
\end{cluster}

\begin{cluster}
\label{grp:grm:ordset-interface::split}

\begin{gram}[split]
\label{grm:ordset-interface::split}
\begin{verbatim}
val split : set * Key.t -> set * bool * set
\end{verbatim}
\sml{split (s, k)} evaluates to $(\ell, m, r)$ where $\ell$ is the set of elements
in $s$ which are smaller than $k$, $r$ is the set of elements in $s$ which
are greater than $k$, and $m$ indicates whether or not $k$ is a member of $s$.

\end{gram}
\end{cluster}

\begin{cluster}
\label{grp:grm:ordset-interface::join}

\begin{gram}[join]
\label{grm:ordset-interface::join}
\begin{verbatim}
val join : set * set -> set
\end{verbatim}
For sets $a$ and $b$ where the largest element of $a$ is strictly smaller than
the smallest element of $b$, \sml{join (a, b)} evaluates to $a \cup b$.
Otherwise it raises \sml{Order}.

\end{gram}
\end{cluster}

\begin{cluster}
\label{grp:grm:ordset-interface::getrange}

\begin{gram}[getRange]
\label{grm:ordset-interface::getrange}
\begin{verbatim}
val getRange : set -> Key.t * Key.t -> set
\end{verbatim}
\sml{getRange s (x, y)} evaluates to the set
$\{ k \in s \mathbin| x \leq k \leq y \}$

\end{gram}
\end{cluster}

\begin{cluster}
\label{grp:grm:ordset-interface::rank}

\begin{gram}[rank]
\label{grm:ordset-interface::rank}
\begin{verbatim}
val rank : set * Key.t -> int
\end{verbatim}
\sml{rank (s, k)} evaluates to $\left| \{ k' \in s \mathbin| k' < k \}\right|$,
the number of elements in $s$ which are strictly smaller than $k$.

\end{gram}
\end{cluster}

\begin{cluster}
\label{grp:grm:ordset-interface::select}

\begin{gram}[select]
\label{grm:ordset-interface::select}
\begin{verbatim}
val select : set * int -> Key.t option
\end{verbatim}
\sml{select (s, i)} returns the $i^\text{th}$ smallest element of $s$, or
\sml{NONE} if either $i < 0$ or $i \geq |s|$.

\end{gram}
\end{cluster}

\begin{cluster}
\label{grp:grm:ordset-interface::splitrank}

\begin{gram}[splitRank]
\label{grm:ordset-interface::splitrank}
\begin{verbatim}
val splitRank : set * int -> set * set
\end{verbatim}
\sml{splitRank (s, i)} evalutes to $(\ell,r)$ where $\ell$ is the set of the
$i$ smallest elements of $s$, and $r$ is the set of the $|s|-i$ largest
elements of $s$. Raises \sml{Fail} if either $i < 0$ or $i \geq |s|$.

\end{gram}
\end{cluster}

