\chapter{Augmented OrdTable Interface}
\label{ch:aug-ordtable-interface}
\begin{preamble}
The \sml{AUG\_ORDTABLE} interface specifies a mapping from keys to values,
augmented with
efficient retrieval of reduced values for ranges of keys. Tables do not have
duplicate
keys, so there is a unique associated value for each key in the domain of
a table. The key type is given by the \sml{Key} substructure. The value type
is given by the \sml{Val} substructure.
\end{preamble}

\begin{note}
The \sml{AUG\_ORDTABLE} signature is nearly identical to \sml{ORDTABLE} except
for the following.
\begin{itemize}
  \item Tables are no longer polymorphic in the type of values associated with
  keys.
  \item There is now a \ref{gr:aug-ordtable-interface:Val} substructure which
  fixes the value type, and specifies how reduced values are computed.
  \item The \ref{gr:aug-ordtable-interface:reduceVal} function retrieves
  reduced values.
  \item The functions \sml{collect} and \sml{iteratePrefixes} have been
  removed, since these rely upon a polymorphic value type.
\end{itemize}
\end{note}

\section{Augmented, Ordered Tables}

\subsection{Summary}

\begin{gram}
\begin{verbatim}
signature AUG_ORDTABLE =
sig
  structure Key : ORDKEY
  structure Val : MONOID
  structure Seq : SEQUENCE

  type t
  type table = t

  structure Set : ORDSET where Key = Key and Seq = Seq

  exception Order

  val size : table -> int
  val domain : table -> Set.t
  val range : table -> Val.t Seq.t
  val toString : table -> string
  val toSeq : table -> (Key.t * Val.t) Seq.t

  val find : table -> Key.t -> Val.t option
  val insert : table * (Key.t * Val.t) -> table
  val insertWith : (Val.t * Val.t -> Val.t) -> table * (Key.t * Val.t) -> table
  val delete : table * Key.t -> table

  val empty : unit -> table
  val singleton : Key.t * Val.t -> table
  val tabulate : (Key.t -> Val.t) -> Set.t -> table
  val fromSeq : (Key.t * Val.t) Seq.t -> table

  val map : (Val.t -> Val.t) -> table -> table
  val mapKey : (Key.t * Val.t -> Val.t) -> table -> table
  val filter : (Val.t -> bool) -> table -> table
  val filterKey : (Key.t * Val.t -> bool) -> table -> table

  val reduce : (Val.t * Val.t -> Val.t) -> Val.t -> table -> Val.t
  val iterate : ('a * Val.t -> 'a) -> 'a -> table -> 'a

  val union : (Val.t * Val.t -> Val.t) -> table * table -> table
  val intersection : (Val.t * Val.t -> Val.t) -> table * table -> table
  val difference : table * table -> table

  val restrict : table * Set.t -> table
  val subtract : table * Set.t -> table

  val $ : (Key.t * Val.t) -> table

  (* ordered table functions *)
  val first : table -> (Key.t * Val.t) option
  val last : table -> (Key.t * Val.t) option

  val prev : table * Key.t -> (Key.t * Val.t) option
  val next : table * Key.t -> (Key.t * Val.t) option

  val split : table * Key.t -> table * Val.t option * table
  val join : table * table -> table

  val getRange : table -> Key.t * Key.t -> table

  val rank : table * Key.t -> int
  val select : table * int -> (Key.t * Val.t) option
  val splitRank : table * int -> table * table

  (* augmentation *)
  val reduceVal : table -> Val.t
end
\end{verbatim}
\end{gram}

\subsection{Substructures}

\begin{gram}[Key]
\begin{verbatim}
structure Key : ORDKEY
\end{verbatim}
The \sml{Key} substructure defines the type of keys in a table, which are
totally ordered according to the provided comparison function.
\end{gram}

\begin{gram}[Val]
\label{gr:aug-ordtable-interface:Val}
\begin{verbatim}
structure Val : MONOID
\end{verbatim}
The \sml{Val} substructure defines the type of values in a table, which
ascribe to \ref{gr:aug-ordtable-interface:MONOID}, specifying how reduced
values are computed.
\end{gram}

\begin{gram}[Seq]
\begin{verbatim}
structure Seq : SEQUENCE
\end{verbatim}
The \sml{Seq} substructure defines the underlying sequence type, so that we
may convert tables to and from sequences.
\end{gram}

\begin{gram}[Set]
\begin{verbatim}
structure Set : ORDSET
\end{verbatim}
The \sml{Set} substructure contains operations on ordered sets with elements of
type \sml{Key.t}.
\end{gram}


\subsection{Types}

\begin{gram}
\begin{verbatim}
type t
type table = t
\end{verbatim}
The abstract table type with keys of type \sml{Key.t} and values of type
\sml{Val.t}. The alias \sml{table} is for readability in the signature.
\end{gram}


\subsection{Exceptions}

\begin{gram}
\begin{verbatim}
exception Order
\end{verbatim}
\sml{Order} is raised when the ordering invariant would be violated.
\end{gram}


\subsection{Values}

\begin{gram}[size]
\begin{verbatim}
val size : table -> int
\end{verbatim}
The number of key-value pairs in a table.
\end{gram}

\begin{gram}[domain]
\begin{verbatim}
val domain : table -> Set.t
\end{verbatim}
Return the set of all keys in a table.
\end{gram}

\begin{gram}[range]
\begin{verbatim}
val range : table -> Val.t Seq.t
\end{verbatim}
Return a sequence of all values in a table. The order of the elements is
implementation-defined.
\end{gram}

\begin{gram}[toString]
\begin{verbatim}
val toString : table -> string
\end{verbatim}
\sml{toString t} returns a string representation of $t$. Each key is converted
to a string via \sml{Key.toString} and each value is converted via
\sml{Val.toString}.
\end{gram}

\begin{gram}[toSeq]
\begin{verbatim}
val toSeq : table -> (Key.t * Val.t) Seq.t
\end{verbatim}
Return a sequence of all key-value pairs in a table, sorted by key.
\end{gram}

\begin{gram}[find]
\begin{verbatim}
val find : table -> Key.t -> Val.t option
\end{verbatim}
\sml{find t k} returns \sml{SOME v} if $(k \mapsto v) \in t$, and \sml{NONE}
otherwise.
\end{gram}

\begin{gram}[insert]
\begin{verbatim}
val insert : table * (Key.t * Val.t) -> table
\end{verbatim}
\sml{insert (t, (k, v))} returns $t \cup \{k \mapsto v\}$. If $k$ is already
in $t$, then the new value $v$ is given precedence. It is logically equivalent
to \sml{insertWith (fn (\_, v) => v) (t, (k, v))}.
\end{gram}

\begin{gram}[insertWith]
\begin{verbatim}
val insertWith : (Val.t * Val.t -> Val.t) -> table * (Key.t * Val.t) -> table
\end{verbatim}
\sml{insertWith f (t, (k, v))} returns $t \cup \{k \mapsto v\}$ if $k$ is not
already a member of $t$, and otherwise it returns $t \cup \{k \mapsto f(v',v)\}$
where $k \mapsto v'$ is already in $t$.
\end{gram}

\begin{gram}[delete]
\begin{verbatim}
val delete : table * Key.t -> table
\end{verbatim}
\sml{delete (t, k)} removes the key $k$ from $t$ only if $k$ is a member of $t$.
That is, if $(k \mapsto v) \in t$ then it returns
$t \setminus \{(k \mapsto v)\}$, otherwise it returns $t$.
\end{gram}

\begin{gram}[empty]
\begin{verbatim}
val empty : unit -> table
\end{verbatim}
Construct the empty table.
\end{gram}

\begin{gram}[singleton]
\label{gr:aug-ordtable-interface:singleton}
\begin{verbatim}
val singleton : Key.t * Val.t -> table
\end{verbatim}
\sml{singleton (k, v)} constructs the singleton table $\{k \mapsto v\}$.
\end{gram}

\begin{gram}[tabulate]
\begin{verbatim}
val tabulate : (Key.t -> Val.t) -> Set.t -> table
\end{verbatim}
\sml{tabulate f s} evaluates to the table $\{(k \mapsto f(k)) : k \in s\}$.
\end{gram}

\begin{gram}[fromSeq]
\begin{verbatim}
val fromSeq : (Key.t * Val.t) Seq.t -> table
\end{verbatim}
Return the table representation of a sequence of key-value pairs. If there are
duplicate keys, then it takes the value associated with the first occurence in
the sequence.
\end{gram}

\begin{gram}[map]
\begin{verbatim}
val map : (Val.t -> Val.t) -> table -> table
\end{verbatim}
\sml{map f t} evaluates to $\{k \mapsto f(v) : (k \mapsto v) \in t\}$.
\end{gram}

\begin{gram}[mapKey]
\begin{verbatim}
val mapKey : (Key.t * Val.t -> Val.t) -> table -> table
\end{verbatim}
\sml{mapKey f t} evaluates to $\{k \mapsto f(k, v) : (k \mapsto v) \in t\}$.
\end{gram}

\begin{gram}[filter]
\begin{verbatim}
val filter : (Val.t -> bool) -> table -> table
\end{verbatim}
\sml{filter p t} produces a table containing all $(k \mapsto v) \in t$ which
satisfies $p(v)$.
\end{gram}

\begin{gram}[filterKey]
\begin{verbatim}
val filterKey : (Key.t * Val.t -> bool) -> table -> table
\end{verbatim}
\sml{filterKey p t} produces a table containing all $(k \mapsto v) \in t$ which
satisfies $p(k,v)$.
\end{gram}

\begin{gram}[reduce]
\begin{verbatim}
val reduce : (Val.t * Val.t -> Val.t) -> Val.t -> table -> Val.t
\end{verbatim}
\sml{reduce f b t} is logically equivalent to \sml{Seq.reduce f b (range t)}.
\end{gram}

\begin{gram}[iterate]
\begin{verbatim}
val iterate : ('a * Val.t -> 'a) -> 'a -> table -> 'a
\end{verbatim}
\sml{iterate f b t} is logically equivalent to \sml{Seq.iterate f b (range t)}.
\end{gram}

\begin{gram}[union]
\begin{verbatim}
val union : (Val.t * Val.t -> Val.t) -> (table * table) -> table
\end{verbatim}
\sml{union f (a, b)} evaluates to $a \cup b$. For keys $k$ where $(k \mapsto v) \in a$
and $(k \mapsto w) \in b$, the result contains $k \mapsto f(v,w)$.
\end{gram}

\begin{gram}[intersection]
\begin{verbatim}
val intersection : (Val.t * Val.t -> Val.t) -> table * table -> table
\end{verbatim}
\sml{intersection f (a, b)} evaluates to $a \cap b$. Every intersecting key
$k$ is mapped to $f(v,w)$ where $(k \mapsto v) \in a$ and $(k \mapsto w) \in b$.
\end{gram}

\begin{gram}[difference]
\begin{verbatim}
val difference : table * table -> table
\end{verbatim}
\sml{difference (a, b)} evaluates to $a \setminus b$. The values in the output
are taken from $a$.
\end{gram}

\begin{gram}[restrict]
\begin{verbatim}
val restrict : table * Set.t -> table
\end{verbatim}
\sml{restrict (t, s)} returns the table of $\{(k \mapsto v) \in t \mathbin| k \in s\}$.
It is therefore essentially an intersection. The name is motivated by the
notion of restricting a function to a smaller domain, where we interpret a table
as a function.
\end{gram}

\begin{gram}[subtract]
\begin{verbatim}
val subtract : table * Set.t -> table
\end{verbatim}
\sml{subtract (t, s)} returns the table of
$\{(k \mapsto v) \in t \mathbin| k \not\in s\}$.
The name is motivated by the notion of a domain subtraction on a function.
\end{gram}

\begin{gram}[\$]
\begin{verbatim}
val $ : (Key.t * Val.t) -> table
\end{verbatim}
An alias for \ref{gr:aug-ordtable-interface:singleton}.
\end{gram}

\begin{gram}[first]
\begin{verbatim}
val first : table -> (Key.t * Val.t) option
\end{verbatim}
Return the smallest key (together with its associated value) of a table, or
\sml{NONE} if the table is empty.
\end{gram}

\begin{gram}[last]
\begin{verbatim}
val last : table -> (Key.t * Val.t) option
\end{verbatim}
Return the largest key (together with its associated value) of a table, or
\sml{NONE} if the table is empty.
\end{gram}

\begin{gram}[prev]
\begin{verbatim}
val prev : table * Key.t -> (Key.t * Val.t) option
\end{verbatim}
\sml{prev (t, k)} returns the largest key (together with its associated value)
in $t$
which is strictly smaller than $k$, or \sml{NONE} if there is no such key.
\end{gram}

\begin{gram}[next]
\begin{verbatim}
val next : table * Key.t -> (Key.t * Val.t) option
\end{verbatim}
\sml{next (t, k)} returns the smallest key (together with its associated value)
in $t$
which is strictly greater than $k$, or \sml{NONE} if there is no such key.
\end{gram}

\begin{gram}[split]
\begin{verbatim}
val split : table * Key.t -> table * Val.t option * table
\end{verbatim}
\sml{split (t, k)} evaluates to $(\ell, v, r)$ where $\ell$ consists of all
key-value pairs with keys strictly smaller than $k$, $r$ consists of all key-value
pairs with keys strictly greater than $k$, and $v$ is $k$'s associated value
in $t$ (or \sml{NONE} if $k$ is not present in $t$).
\end{gram}

\begin{gram}[join]
\begin{verbatim}
val join : table * table -> table
\end{verbatim}
Given tables $a$ and $b$ where all keys in $a$ are strictly smaller than all
keys in $b$, \sml{join (a, b)} evaluates to $a \cup b$. Otherwise, it raises
\sml{Order}.
\end{gram}

\begin{gram}[getRange]
\begin{verbatim}
val getRange : table -> Key.t * Key.t -> table
\end{verbatim}
\sml{getRange t (x, y)} evaluates to $\{(k \mapsto v) \in t \mathbin| x \leq k \leq y\}$
\end{gram}

\begin{gram}[rank]
\begin{verbatim}
val rank : table * Key.t -> int
\end{verbatim}
\sml{rank (t, k)} evaluates to the number of keys in $t$ which are strictly
smaller than $k$.
\end{gram}

\begin{gram}[select]
\begin{verbatim}
val select : table * int -> (Key.t * Val.t) option
\end{verbatim}
\sml{select (t, i)} evaluates to the $i^\text{th}$ smallest key (and its
associated value), or \sml{NONE} if either $i < 0$ or $i \geq |t|$.
\end{gram}

\begin{gram}[splitRank]
\begin{verbatim}
val splitRank : table * int -> table * table
\end{verbatim}
\sml{splitRank (t, i)} evalutes to $(\ell,r)$ where $\ell$ consists of the $i$
keys of $t$, and $r$ is the set of the $|t|-i$ largest
keys of $t$. Raises \sml{Fail} if either $i < 0$ or $i \geq |t|$.
\end{gram}

\begin{gram}[reduceVal]
\label{gr:aug-ordtable-interface:reduceVal}
\begin{verbatim}
val reduceVal : table -> Val.t
\end{verbatim}
\sml{reduceVal t} is logically equivalent to
\sml{Seq.reduce Val.f Val.I (range t)}. See
\ref{gr:aug-ordtable-interface:MONOID} below for more information.
\end{gram}


\section{Reduced Values as Monoids}

\subsection{Summary}

\begin{gram}
The value type of an augmented ordered table ascribes to \sml{MONOID}, which
specifies an abstract type $t$ together with an identity $I$, an
\emph{associative} combination function $f$, and a \sml{toString} function for
convenience. The reduced value of a table is given by ``summing'' all values in
the table, with respect to $f$.
\end{gram}

\begin{gram}[MONOID]
\label{gr:aug-ordtable-interface:MONOID}
\begin{verbatim}
signature MONOID =
sig
  type t
  val I : t
  val f : t * t -> t
  val toString : t -> string
end
\end{verbatim}
\end{gram}

\subsection{Types and Values}

\begin{gram}
\begin{verbatim}
type t
\end{verbatim}
The abstract type of the monoid.
\end{gram}

\begin{gram}[I]
\begin{verbatim}
val I : t
\end{verbatim}
The identity of the monoid. Must satisfy $f(I, x) = x = f(x, I)$ for all
monoid elements $x$.
\end{gram}

\begin{gram}[f]
\begin{verbatim}
val f : t * t -> t
\end{verbatim}
The associative combination function of the monoid.
\end{gram}

\begin{gram}[toString]
\begin{verbatim}
val toString : t -> string
\end{verbatim}
Return a string representation.
\end{gram}

