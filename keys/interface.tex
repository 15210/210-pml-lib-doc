\chapter{Key Interfaces}
\label{ch:key-interface}
\begin{preamble}
The \sml{EQKEY}, \sml{ORDKEY}, and \sml{HASHKEY} signatures specify abstract
key types which support various forms of equality testing and comparison.
\end{preamble}

\section{Summary}

\begin{gram}
The \sml{EQKEY} signature specifies a key which can only be tested for
equality.
\begin{verbatim}
signature EQKEY =
sig
  type t
  val equal : t * t -> bool
  val toString : t -> string
end
\end{verbatim}
\end{gram}

\begin{gram}
The \sml{ORDKEY} signature specifies a key which, in addition to supporting
equality, is also totally ordered. Any structure which ascribes to \sml{ORKEY}
also implicitly ascribes to \sml{EQKEY}.
\begin{verbatim}
signature ORDKEY =
sig
  include EQKEY
  val compare : t * t -> order
end
\end{verbatim}
\end{gram}

\begin{gram}
The \sml{HASHKEY} signature specifies a key which additionally offers a
pseudo-random hashing operation. Any structure which ascribes to \sml{HASHKEY}
also implicitly ascribes to both \sml{ORDKEY} and \sml{EQKEY}.
\begin{verbatim}
signature HASHKEY =
sig
  include ORDKEY
  val hash : t -> int
end
\end{verbatim}
\end{gram}

\section{Types and Values}

\begin{gram}
\begin{verbatim}
type t
\end{verbatim}
The abstract key type.
\end{gram}

\begin{gram}[equal]
\begin{verbatim}
val equal : t * t -> bool
\end{verbatim}
Determine whether or not the arguments are considered equal. This function is
reflexive, symmetric, and transitive.
\end{gram}

\begin{gram}[toString]
\begin{verbatim}
val toString : t -> string
\end{verbatim}
Return a string representation of the key.
\end{gram}

\begin{gram}[compare]
\begin{verbatim}
val compare : t * t -> order
\end{verbatim}
Return one of \sml{LESS}, \sml{EQUAL}, or \sml{GREATER} as appropriate for
the argument pair. This operation is transitive. It is also consistent:
\sml{compare(x,y) = EQUAL} if and only if \sml{equal(x,y)}, and
\sml{compare(x,y) = LESS} if and only if \sml{compare(y,x) = GREATER}.
\end{gram}

\begin{gram}[hash]
\begin{verbatim}
val hash : t -> int
\end{verbatim}
Return a pseudo-random hash.
\end{gram}

